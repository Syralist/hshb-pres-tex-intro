\documentclass{beamer}

\usepackage[T1]{fontenc}
\usepackage[ngerman]{babel}
\usepackage{selinput}
\SelectInputMappings{
   adieresis={ä},
   germandbls={ß}
   }

\usepackage{listings}
\lstset{language=[LaTeX]TeX,
escapeinside=||}

\usetheme{Hannover}  %% Themenwahl

\setbeamercovered{transparent}
%\setbeamertemplate{footline}[frame number]
 
\title{\LaTeX - Eine Einführung}
\author{Thomas Helmke}
\date{xx.12.2013}
 
\begin{document}
\maketitle
\frame{\tableofcontents}
 
\section{Was ist \LaTeX?}
\begin{frame}[<+->] %%Eine Folie
	\frametitle{Was ist \LaTeX?} %%Folientitel
	\begin{itemize}
		\item Textsatzsystem
		\item Auszeichnungssprache (ähnlich wie HTML)
		\item ...
	\end{itemize}
\end{frame}
\begin{frame}[<+->] %%Eine Folie
	\frametitle{Unterschied zu Word und Co.} %%Folientitel
	\begin{itemize}
		\item Form und Funktion getrennt
		\item Dokument wird kompiliert
		\item Erzeugt typographisch gute Seiten 
	\end{itemize}
\end{frame}

\setbeamercovered{}
\section{Aufbau eines Dokuments}
\subsection{Kopf}
\begin{frame}[fragile] %%Eine Folie
	\frametitle{Kopf eines \LaTeX-Dokuments} %%Folientitel
	\begin{lstlisting}
	\documentclass{scrartcl}|\pause|
	
	\usepackage{selinput}
	\SelectInputMappings{
		   adieresis={|ä|},
		   germandbls={|ß|}
		   }|\pause|
		   
	\usepackage[T1]{fontenc}
	\usepackage{lmodern}|\pause|
	
	\usepackage[ngerman]{babel}
	\end{lstlisting}
\end{frame}
\subsection{Titelei}
\begin{frame}[fragile] %%Eine Folie
	\frametitle{Die Titelei} %%Folientitel
	\begin{lstlisting}
	\subject{Vortrag}
	\title{Beispiel Dokument}
	\subtitle{Mit Untertitel}|\pause|
	
	\author{Thomas Helmke}
	\date{2014-05-06}|\pause|
	
	\publishers{Hackerspace Bremen e.V.}
	\end{lstlisting}
\end{frame}
\subsection{Struktur}
\begin{frame}[fragile] %%Eine Folie
	\frametitle{Struktur eines Dokuments} %%Folientitel
	\begin{lstlisting}
	\begin{document}|\pause|
	
	\maketitle|\pause|
	
	\section{Erster Abschnitt}
	\subsection{Ein Unterabschnitt}
	Text.|\pause|
	
	\end{document}
	\end{lstlisting}
\end{frame}

\end{document}
