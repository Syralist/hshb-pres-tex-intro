\documentclass[svgnames]{beamer}

\usepackage{lmodern}
\usepackage[T1]{fontenc}
\usepackage[ngerman]{babel}
\usepackage{selinput}
\SelectInputMappings{
   adieresis={ä},
   germandbls={ß}
   }

\usepackage{listings}
\lstset{language=[LaTeX]TeX,
escapeinside=||,
frame=single,
backgroundcolor=\color{Cornsilk}}

\usetheme{Hannover}  %% Themenwahl

\setbeamercovered{transparent}
%\setbeamertemplate{footline}[frame number]
\usecolortheme{spruce}		% grün
\usecolortheme[named=MSUgreen]{structure}
\newcommand{\hshblogo}{\includegraphics[width=1.3cm]{space-logo}}
 
\title[\hshblogo]{\LaTeX - Eine Einführung}
\author{Thomas Helmke}
\date{06.05.2014}
\logo{\includegraphics{space-logo}}
 
\begin{document}
\maketitle
\frame{\tableofcontents}
 
\section{Was ist \LaTeX?}
\begin{frame}[<+->] %%Eine Folie
	\frametitle{Was ist \LaTeX?} %%Folientitel
	\begin{itemize}
		\item Textsatzsystem
		\item Auszeichnungssprache (ähnlich wie HTML)
		\item Seitenaufbau nach typographischen Grundsätzen
	\end{itemize}
\end{frame}
\begin{frame}[<+->] %%Eine Folie
	\frametitle{Unterschied zu Word und Co.} %%Folientitel
	\begin{itemize}
		\item Form und Funktion getrennt
		\item Dokument wird kompiliert
		\item What you see is what you mean.
	\end{itemize}
\end{frame}

\setbeamercovered{}
\section{Vor dem Schreiben}
\subsection{Dokumenten\-klasse}
\begin{frame}[fragile] %%Eine Folie
	\frametitle{Dokumentenklasse} %%Folientitel
	\begin{columns}
	\column{.67\textwidth}
		\begin{itemize}[<+->]
			\item Drei Grundklassen:
			\begin{itemize}[<+->]
				\item article \\(Aufsätze, wenige Abschnitte)
				\item report \\(Berichte, mehrere Kapitel)
				\item book \\(Bücher, mehrteilig mit vielen Kapiteln)
			\end{itemize}
		\end{itemize}
	\column{.33\textwidth}
		\pause
		\begin{itemize}
			\item KOMA Script:
			\begin{itemize}
				\item scrartcl
				\item[] 
				\item scrreprt
				\item[] 
				\item scrbook
				\item[] 
				\item[]
			\end{itemize}
		\end{itemize}
	\end{columns}
	\pause
	\vspace*{\baselineskip}
	\begin{lstlisting}
	\documentclass{scrartcl}
	\end{lstlisting}
\end{frame}
\subsection{Zeichensatz}
\begin{frame}[fragile] %%Eine Folie
	\frametitle{Zeichensätze} %%Folientitel
	\begin{itemize}
		\item Zeichensatz muss bekannt gemacht werden\pause
		\item Paket \lstinline|inputenc|\pause
		\item Optionen \lstinline|utf-8, latin1, ansinew|\pause
		\item Einfacher mit \lstinline|selinput|\pause
	\end{itemize}
	\vspace*{\baselineskip}
	\begin{lstlisting}
	\usepackage{selinput}
	\SelectInputMappings{
		adieresis={|ä|},
		germandbls={|ß|}
		}
	\end{lstlisting}
\end{frame}
\subsection{Schriftart}
\begin{frame}[fragile] %%Eine Folie
	\frametitle{Schriftarten} %%Folientitel
	\begin{itemize}
		\item Standardschrift \lstinline|Computer Modern|\pause
		\item Erweiterte Version \lstinline|Latin Modern|\pause
		\item Europäische Akzente mit T1 Encoding\pause
		\item Weitere Schriften als Pakete \url{http://www.tug.dk/FontCatalogue/}\pause
	\end{itemize}
	\vspace*{\baselineskip}
	\begin{lstlisting}
	\usepackage{lmodern}
	\usepackage[T1]{fontenc}
	\end{lstlisting}
\end{frame}
\subsection{Sprache}
\begin{frame}[fragile] %%Eine Folie
	\frametitle{Sprachen} %%Folientitel
	\begin{itemize}
		\item Sprachunterstützung durch \lstinline|babel|\pause
		\item Lädt: Silbentrennung, Bezeichner\pause
		\item Weitergabe der Sprache an weitere Pakete\pause
	\end{itemize}
	\vspace*{\baselineskip}
	\begin{lstlisting}
	\usepackage[ngerman]{babel}
	\end{lstlisting}
\end{frame}

\section{Der erste Inhalt}
\subsection{Titelei}
\begin{frame}[fragile] %%Eine Folie
	\frametitle{Die Titelei} %%Folientitel
	\begin{lstlisting}
	\subject{Vortrag}
	\title{Beispiel Dokument}
	\subtitle{Mit Untertitel}|\pause|
	
	\author{Thomas Helmke}
	\date{2014-05-06}|\pause|
	
	\publishers{Hackerspace Bremen e.V.}
	\end{lstlisting}
\end{frame}
\subsection{Struktur}
\begin{frame}[fragile] %%Eine Folie
	\frametitle{Struktur eines Dokuments} %%Folientitel
	\begin{lstlisting}
	\begin{document}|\pause|
	
	\maketitle|\pause|
	
	\section{Erster Abschnitt}
	\subsection{Ein Unterabschnitt}
	Text.|\pause|
	
	\end{document}
	\end{lstlisting}
\end{frame}
\begin{frame}[fragile] %%Eine Folie
	\frametitle{Umbrüche} %%Folientitel
	\begin{lstlisting}
	\begin{document}
	
	\maketitle
	
	\section{Erster Abschnitt}
	\subsection{Ein Unterabschnitt}
	Zeilenumbruch (b|ö|se):\\|\pause|
	Absatz:
	
	Getrennt durch Leerzeile.|\pause|
	Seitenumbruch: \clearpage
	
	\end{document}
	\end{lstlisting}
\end{frame}

\section{Weitere Techniken}
\subsection{Graphiken}
\begin{frame}[fragile] %%Eine Folie
	\frametitle{Graphik einbinden} %%Folientitel
	\begin{itemize}[<+->]
		\item Bilder als JPG, PNG, PDF
		\item Paket \lstinline|graphicx|
		\item Einfache Variante
			\begin{lstlisting}
			\includegraphics{bilddatei}
			\end{lstlisting}
		\item Bessere Methode
			\begin{lstlisting}
			\begin{figure}[htb]
				\includegraphics{bilddatei}
				\label{fig:Bild}
				\caption{Ein Bild}
			\end{figure}
			\end{lstlisting}
	\end{itemize}
\end{frame}
\subsection{Tabellen}
\begin{frame}[fragile] %%Eine Folie
	\frametitle{Tabellen setzen} %%Folientitel
	\begin{itemize}[<+->]
		\item Tabelle definieren
			\begin{lstlisting}
			\begin{tabular}{ccc}
			a & b & c\\
			d & e & f
			\end{tabular}
			\end{lstlisting}
		\item Tabelle einbinden
			\begin{lstlisting}
			\begin{table}[htb]
			\caption{Eine Tabelle}
			\label{tab:tabelle1}
			\begin{tabular}{ccc}
			a & b & c\\
			d & e & f
			\end{tabular}
			\end{table}
			\end{lstlisting}
	\end{itemize}
\end{frame}
\begin{frame}[fragile] %%Eine Folie
	\frametitle{Tabellen formatieren} %%Folientitel
	Einfach, aber hässlich
		\begin{lstlisting}[escapeinside=@@]
		\begin{tabular}{|c|c|c|}
		\hline
		a & b & c\\
		\hline
		d & e & f\\
		\hline
		\end{tabular}
		\end{lstlisting}
\end{frame}
\begin{frame}[fragile] %%Eine Folie
	\frametitle{Tabellen formatieren} %%Folientitel
	Besser mit \lstinline|booktabs|
		\begin{lstlisting}
		\begin{tabular}{ccc}
			\toprule
			a & b & c\\
			\midrule
			d & e & f\\
			g & h & i\\
			\bottomrule
		\end{tabular}
		\end{lstlisting}
\end{frame}
\subsection{Verweise}
\begin{frame}[fragile] %%Eine Folie
	\frametitle{Verweise} %%Folientitel
	\begin{lstlisting}
	\section{Erster Abschnitt}
	\label{sec:Kapitel1}
	\subsection{Ein Unterabschnitt}
	Text mit Verweis: \ref{sec:Kapitel1}
	Verweis auf Seite: \pageref{sec:Kapitel1}
	\end{lstlisting}
\end{frame}
\begin{frame}[fragile] %%Eine Folie
	\frametitle{Verweise in hübsch} %%Folientitel
	\begin{lstlisting}
	\usepackage[german]{fancyref}
	\usepackage[colorlinks=true]{hyperref}
	
	...
	
	\section{Erster Abschnitt}
	\label{sec:Kapitel1}
	\subsection{Ein Unterabschnitt}
	Verweis auf Kapitel und Seite: 
	\fref{sec:Kapitel1}
	\end{lstlisting}
\end{frame}

\end{document}
